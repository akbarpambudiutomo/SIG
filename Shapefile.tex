% Tugas ke 2 SIG Kelompok 4
% Akbar Pambudi Utomo (1154094)
% Julham Ramadhana (1154069)


/section{Shapefile}
/subsection{Pengertian Shapefile}
Shapefile ArcView memiliki format data tersendiri yang disebut dengan shapefiles. 
Shapefiles adalah format data yang menyimpan lokasi geometrik dan informasi atribut dari suatu feature geografis. 
Pada umumnya kita hanya butuh satu file kerja seperti file Microsoft Word dengan extension file *.doc, 
akan tetapi shapefile memiliki perbedaan, yaitu bahwa satu shapefile memiliki beberapa file yang saling berkaitan satu sama lainnya. 
Beberapa file ini memiliki extension yang 42 berbeda-beda yang disimpan dalam workspace yang sama.
Catatan : tiga file extension pertama adalah bagian file extension yang harus ada dalam sebuah shapefile, file extension berikutnya sifatnya optional.
Pada ArcGIS fitur geografis di shapefile dapat ditunjukkan oleh titik, garis, atau poligon (area). Ruang kerja yang berisikan shapefile mungkin juga berisi dBase, yang dapat menyimpan atribut tambahan yang dapat digabungkan ke fitur shapefile. Semua file yang memiliki ekstensi seperti file .text, .asc, .csv, atau tab muncul di ArcCatalog sebagai file text secara default. akan tetapi, pada kotak dialog Opsi Kita dapat memilih tipe file mana yang harus direpresentasikan sebagai file teks dan seharusnya tidak ditampilkan di pohon Catalog. Ketika file teks berisi nilai koma dan tab-delimited, kita bisa melihat isi file di tampilan table ArcCatalog dan menggabungkannya ke dalam fitur geografis. file teks bisa juga kita hapus, tetapi isinya hanya bisa dibaca di ArcCatalog.
